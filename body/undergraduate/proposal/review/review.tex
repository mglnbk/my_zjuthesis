\cleardoublepage
\newrefsection
\chapter{文献综述}

\section{背景介绍}

\subsection{药物研发的挑战}
在近几十年,人们对疾病分子机制的认识和科学技术的发展为开发研究新药
(Pharmaceutical Research and Development, R\&D)
提供了广阔的开发空间\cite{pammolliProductivityCrisisPharmaceutical2011}。一些在过去昂贵的药物在如今得益于如今的生物技术的发展已经不再是望而生畏,
最典型的比如化疗药物紫杉醇类、吉西他滨、奥沙利铂、表柔比星等。然而现如今世界范围的生物医药公司都面临着一系列棘手的难题:
药物研发的高损耗率、高时间成本以及多变的药物准入规则等等,这些都导致了药物研发的极高的成本\cite{pushpakomDrugRepurposingProgress2019}。

\par 以药物研发的高损耗率为例,降低临床试验药物损耗率问题一直是生物医药公司的目标与挑战。对生物医药产品的上市前的临床试验通常分为四个阶段,一期、二期、三期和四期,
常规的药物开发流程都需要完整经历这四个流程。每一个阶段对于药物的审核都有具体的标准,不符合标准的候选药物会被终止研究,从而导致了
药物研发损耗。高损耗率可以归咎于药物本身的药物性质、药物效能以及药物安全性,在最近十年因为后两者导致的药物损耗逐渐升高\cite{bunnageGettingPharmaceuticalBack2011, kolaCanPharmaceuticalIndustry2004, hayClinicalDevelopmentSuccess2014}。
在一项对阿斯利康、礼来、葛兰素史克以及辉瑞四大生物制药公司
从2000年至2010年的药物研发项目数据集研究中,研究人员指出药物的毒性和安全性考量是高损耗率的主要来源\cite{waringAnalysisAttritionDrug2015}。


\par 然而这些高损耗带来的高成本却无法带来足够数量的创新药物开发,
从1990年到2010年,研究人员从药物开发项目数据库发现,研究和开发新药的效率在这二十年呈现出下降趋势
\cite{pammolliProductivityCrisisPharmaceutical2011}。
简单地来说,研发生产效率可以被定义为一种新药所创造的价值与产生该药所需的投资之间的关系\cite{paulHowImproveProductivity2010},这样
药物研发可以简化为一个输入输出模型。在生产方面,如今药物研发成本使得生物制药产业面临着严重的“生产效率危机(Productivity Crisis)”,
如何以更高效率研发新药亟待解决。输出方面制药公司也面临着严峻的考验,过去利润巨大的专利失效\cite{EvaluatePharmaAlphaWorld2014}使得公司无法承担起
新药开发与日俱增的资金投入需求。国家政府对医疗产业投入的紧缩和市场准入门槛的提高\cite{paulHowImproveProductivity2010}同样也对创新药物研发
造成了严重阻碍。因此,如何提高药物研发效率在如今是一个非常重要的议题。


\subsection{药物重定位}
\par 为了解决R\&D的高时间成本和物质成本的问题,很多新型药物开发策略应运而生,其中就包括了药物重定位。
药物重定位是一种为现有药物识别和发展新的适应征或适用人群的一种药物复用策略\cite{ashburnDrugRepositioningIdentifying2004}。
关于药物重定位研究不仅可以拓展现有药物的治疗指征,而且相比较于新型药物设计也可以大大节省时间成本和物质成本。
举例来说,那些正在开发或者已经上市售卖的药物都可以成为非常好重定位候选药物,药物重定位允许他们重新对不同的疾病目标
起到效用,从而减少了药物再开发的成本。重新寻找这些药物的新用途不仅仅会最大化药物本身所带来的收益,并且也可以
惠及病人以及医药产业。

\par 历史上,关于旧药物的复用一度是比较困难的,一些旧药新用的发现依赖于运气和机缘\cite{bolgarDrugRepositioningTreatment2013},
还有一些则是由于人们对药物分子机制有了更深的了解\cite{richShorttermSafetyEfficacy2006}。总之,在过去药物重定位的作用非常有限。
幸运的是,如今信息技术和分子技术的发展使得这一领域重新抓住了发展机遇。新式技术催生了多种多样的药物重定位方法,它们各自基于不同的理论假设,
但是可以大致可以分为两类,即基于计算的方法和基于实验的方法。

\subsubsection{基于实验的药物重定位方法}
基于实验的方法主要可以分为结合亲和力分析和表型筛查两个主要方向\cite{dhirDrugRepurposingOrphan2020}。
对于前者,研究者通常利用色谱质谱法去分析候选药物的结合靶向位点,从而筛选出药物可能的结合位点实现药物重定位的目的。
对于后者表型筛查,在药物发现和开发领域,通常用于指代为确定与疾病直接或间接相关的药物生物学效应而采取的方法\cite{dhirDrugRepurposingOrphan2020}。
其包括很多具体的生化技术,如细胞高通量筛选技术(对单细胞进行筛选药物有效性筛选等)\cite{iljinHighThroughputCellBasedScreening2009, aulnerNextgenerationPhenotypicScreening2019}。
总的来说,在近几十年这些方法越来越多地转向基于表型方法的分子靶点,这在很大程度上依赖于对药物和疾病的分子靶点及相关机制假说\cite{dhirDrugRepurposingOrphan2020}。


\subsubsection{基于计算的药物重定位方法}

\par 不同于基于实验的方法,基于计算的方法大方向上都是属于数据驱动的计算机建模方法,常常根据一系列的公共数据库或者自建队列,对任意种类的数据
进行系统性分析(基因表达数据、药物化学结构、基因组学、蛋白质组学数据以及电子病历数据)
\cite{pushpakomDrugRepurposingProgress2019}。在目前的大数据时代,计算驱动方法是一种有效且快速的药物重定位方法,
依赖于数据的数量和质量。传统方法则更多的聚焦于发掘药物的效能及作用机制相似性或者从药典中筛选有效的靶向药物
\cite{keiserPredictingNewMolecular2009, iorioDiscoveryDrugMode2010}。
但是随着分子生物技术的飞速发展,诸如高通量分子测序等技术产生了海量的生化数据,这使得传统的数据处理方法变得不再适用
\cite{pushpakomDrugRepurposingProgress2019},高性能分布式计算的发展也催生出了更加现代的计算方法,包括机器学习和
网络分析等等\cite{liSurveyCurrentTrends2016}。

\par 这些生物医学数据根据研究对象可以大体上分为药物(Drug)数据、疾病(Disease)、基因(Gene)以及关系(Relationship)
,因此,基于计算的药物重定位可以分为基于药物本身的方法(Drug-based)和基于疾病本身的方法(Disease-based)\cite{jaradaReviewComputationalDrug2020}。
进一步地我们还可以根据方法本身基于的假设进行更加系统性的分类:
\begin{itemize}
    \item \textbf{基于药物本身的方法}: 对于两类药物$R_1$, $R_2$,如果两类药物结构相似,而$R_1$针对于疾病$D_1$的指征,故药物$R_2$也可以治疗$D_1$的指征。
    \item \textbf{基于疾病本身的方法}: 对于两类疾病$D_1$, $D_2$,如果两类疾病表征相似,而药物$R$可以治疗$D_1$的指征,故药物$D$也可以治疗$D_2$的指征。
    \item \textbf{基于基因本身的方法}: 对于两类基因遗传信息$G_1$, $G_2$,如果$G_1$和$G_2$如果含有相同的药物$R$靶向的遗传信息,那么$R$可以同时对$G_1$和$G_2$起作用。
\end{itemize}

\par 基于计算的药物重定位方法遵循着基本的工作流程:
1. 制定研究的策略,即根据具体数据类型来决定研究主题,比如是基于疾病还是基因型。
2. 选择药物重定位具体方法,比如是采取深度学习的方法还是统计遗传学方法等。
3. 对实验结果做出评价,如药物复用的准确率。
4. 最后利用模型和方法对药物或者疾病等进行分析预测。

\section{国内外研究现状}

\subsection{研究方向及进展}
\subsubsection{国内关于药物重定位的发展}
\par 国内关于药物重定位领域的研究按研究方法和数据可分为2个大类6个热点,
一类是基于试验数据的药物 知识发现,另一类是基于科学数据的药物知识挖掘。前者基于临床试验,多以药物分子和细胞受体的作用关 系研究为主,通过建立临床模型发现药物的潜在作用。
后者基于计算机技术,多以科学数据间的相关关系为研究对象,通过计算机构建数据模型进行药物知识发现。
在该领域研究的萌芽期多以中药知识发现为主,发展期则开始转向抗癌药物研究,至稳定期,研究进一步转向更微观的层面\cite{TaiYangFangGuoNeiYaoWuChongDingWeiYanJiuDeZhuTiJiFaZhanMaiLuoFenXi2020}。

\par 总体来看,国内关于该领域的研究起步缓慢,但近几年发展迅速,相关研究主要基于计算机技术和临床试验,
前者偏向于预测,而后者偏向于验证。研究落脚点从宏观向微观进行转变,计算机技术和临床试验也进行了有机的结合\cite{TaiYangFangGuoNeiYaoWuChongDingWeiYanJiuDeZhuTiJiFaZhanMaiLuoFenXi2020}。
\subsubsection{国外关于药物重定位的发展}
国外药物重定位的发展是早于国内的,在早期发展之初常常依赖于运气和巧合:一旦一种药物被发现存在有别于原靶向效应或者
存在非靶向效应,那么就会将该药物进行商用研发\cite{pushpakomDrugRepurposingProgress2019}。
目前最成功的药物复用例子包括了利用枸橼酸西地那非治疗勃起功能障碍依赖于回顾性临床经验,
而在1964年将沙利度胺用于治疗麻风结节性红斑(ENL)\cite{ashburnDrugRepositioningIdentifying2004}和多发性骨髓瘤
\cite{singhalAntitumorActivityThalidomide1999}则基于偶然发现。
在此之后,关于药物重定位的研究迅速发展起来,促进开发更系统的方法来识别可重复利用药物。这些药物包括了
一些处于临床试验后期阶段的药物和治疗罕见病的药物\cite{pushpakomDrugRepurposingProgress2019}。
除了较为传统的药物重定位方法,如GWAS,目前的药物重定位随着计算力的增长开始偏向于大数据分析。除了海量
的测序数据之外,还有越来越多的临床数据可供使用,比如电子健康记录数据(EHR)、临床试验数据及各种Biobank。
\par 另外一个值得注意的一个趋势是深度学习的崛起。2021年Deepmind公司发布了深度学习蛋白质预测模型
AlphaFold\cite{jumperHighlyAccurateProtein2021}引起了轩然大波。除了AlphaFold之外,还有其他
蛋白质模型,如David Baker实验室的Rosetta模型。这些蛋白质预测模型为基于计算的重定位方法探明了更广阔的
发展空间。故此,如今利用大数据技术来进行重定位研究是一个非常具有潜力的发展方向。


\subsection{存在问题}

\subsubsection{数据的复杂性和异质性}
在基于计算的药物重定位方法中,一个关键的部分是关于数据的处理问题。在大数据应用中数据至关重要,然而目前我们可以
利用的数据里面存在着相当多的问题,首先是数据的异质性\cite{gligorijevicIntegrativeMethodsAnalyzing2016, chenIBMWatsonHow2016}。
生物医学数据包括了包括RNA转录组数据为代表的测序数据,电子健康记录数据以及蛋白质结构数据等一系列的不同来源数据。
结合不同来源的数据可以进一步提高分析的水平,但是同时也带来了如何结合这些数据这一难题。
医疗产业的进一步发展也产生了越来越多的非结构化图像数据,比如肺CT影像等。然而在药物重定位
领域对数据集成的研究却非常有限\cite{napolitanoDrugRepositioningMachinelearning2013}。
我们迫切需要能够结合异质数据并整合、分析和解释的技术解决方案。

\subsubsection{公共数据来源有限}
普遍地,基于计算的重定位方法都会利用到海量的生物医学数据,而对于没有建立相应队列的普通研究者来说,
所能利用的数据来源只有目前已经开源的数据库。虽然当前已经公开的转录组数据数据库非常普遍,如Broad研究所
主持的cMAP(Connectivity Mapping)以及ENCODE等一系列数据库,他们同时也包含了标准化的数据,
但对于其他类型的数据,如临床试验数据和分子化学结构、体外实验或影像数据,这样的数据库却很少\cite{pushpakomDrugRepurposingProgress2019}。
而基于机器学习算法的重定位方法则非常依赖于数据的数量与质量,所以优质公共数据库的稀少也是目前值得努力
的方向。

\section{研究展望}
对药物重定位的研究我们希望在未来可以解决这样几个问题。

\subsection{多样化数据库的建立}
在计算集群高速发展的今天,我们对于数据库的需求已经大大超出了以前。而目前即便有着一些高质量数据库,我们如今仍缺少权限去访问公开的II期至IV期的临床试验数据。这些数据可以为科学家提供丰富的重定位研究机会\cite{pushpakomDrugRepurposingProgress2019}。

\subsection{建立更完善的技术体系}
就像之前所提到的关于数据的复杂性和异质性的讨论,目前我们还缺乏一个成熟的技术体系来处理生物医学数据集成的问题。现在随着深度学习技术的进步,也产生了很多新技术如Transformer和蛋白质预测模型。这些方法都具有非常强的数据适应性和拟合能力,但是目前研究对前沿的计算机技术利用仍存在不足,值得进一步发展。

\subsection{完善药物重定位资助和激励制度}
政府科研机构需要为药物重定位相关研究提供更多的资助,包括资助技术,
支持化合物创新和分享数据库访问权限。尤其还需要为罕见疾病的药物
重定位研究提供新的资金来源,如企业家的投资等。最后,我们仍需要采取
措施激励重定位研究,特别是解决专利和监管问题。这些措施可以
包括提供更长的数据访问权限,妥善安排一般公司的专利使用费或通过其他法则条款变化,以确保有足够的空间来获得对科研的投资回报\cite{pushpakomDrugRepurposingProgress2019}。


\newpage
\begingroup
    \linespreadsingle{}
    \printbibliography[title={参考文献}]
\endgroup
