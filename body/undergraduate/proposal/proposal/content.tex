\section{项目的主要内容和技术路线}

\subsection{主要研究内容}
本项目主要研究内容包括对基因型数据的清洗、对药物结构,化学等性质数据的清洗,如果必要的话
对输入数据进行特征工程处理。
对不同的数据类型输入采取不同的处理方式利用数据集成技术对不同来源的数据特征进行归一输出
,形成最后的特征向量。在这之后利用深度神经网络对药物的有效性做出预测。

考虑到数据的稀缺性和数量要求,对于模型的输入内容采取单细胞数据,而病人的具体数据则用于验证。


\subsection{技术路线}
\subsubsection{数据收集和预处理}
依托于单细胞数据库CDRP和GDSC,收集细胞系的\textbf{基因表达数据,拷贝数变异数据,
表观遗传学数据(甲基化水平)以及体细胞突变数据},除此之外也收集关于细胞的药物有效性数据。
这些数据大致呈现为数据矩阵或者索引的形式,故此容易被具象为神经网络用于输入的向量。
在收集数据过程中应该注意尽量保证收集到数据的同时也应该去收集元数据,对数据有一个清晰的了解。
再进行模型建立前,需要对数据本身进行可视化,对连续型变量需要进行离散化的应该选取适合的标准。
如果需要进行归一化的数据需要对其进行必要的预处理。

\subsubsection{模型建立和训练过程}
模型建立使用python编程语言、tensorflow和keras深度学习编程框架。其主要难点在于数据的预处理和集成,
本项目计划使用简单的多层感知机模型,中间通过Dropout层防止过拟合。在设计感知机的神经网络层数的时候,需
进行多次实验之后才能确定每层的神经元数量。对于每层网络的激活函数采取目前效果最好的relu激活函数,在训练过程中
梯度下降采用Adam策略。

\subsubsection{模型评价与测试}
对模型的评价使用AUC指标()



\subsection{可行性分析}

\subsubsection{数据库分析}
单细胞数据库CDRP和GDSC是目前学术界比较成熟的两个数据库,收录了


\subsubsection{导师资源}
毕业设计老师是


\subsubsection{个人能力自述}
