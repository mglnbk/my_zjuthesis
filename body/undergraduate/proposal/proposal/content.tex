\section{项目的主要内容和技术路线}

\subsection{主要研究内容}
本项目主要研究内容包括对基因型数据的清洗、对药物结构,化学等性质数据的清洗,如果必要的话
对输入数据进行特征工程处理。
对不同的数据类型输入采取不同的处理方式利用数据集成技术对不同来源的数据特征进行归一输出
,形成最后的特征向量。在这之后利用深度神经网络对药物的有效性做出预测。

考虑到数据的稀缺性和数量要求,对于模型的输入内容采取单细胞数据,而病人的具体数据则用于验证。


\subsection{技术路线}
\subsubsection{数据收集和预处理}
依托于单细胞数据库CDRP和GDSC,收集细胞系的\textbf{基因表达数据,拷贝数变异数据,
表观遗传学数据(甲基化水平)以及体细胞突变数据},除此之外也收集关于细胞的药物有效性数据。
这些数据大致呈现为数据矩阵或者索引的形式,故此容易被具象为神经网络用于输入的向量。
在收集数据过程中应该注意尽量保证收集到数据的同时也应该去收集元数据,对数据有一个清晰的了解。
再进行模型建立前,需要对数据本身进行可视化,对连续型变量需要进行离散化的应该选取适合的标准。
如果需要进行归一化的数据需要对其进行必要的预处理。

\subsubsection{模型建立和训练过程}
模型建立使用python编程语言、tensorflow和keras深度学习编程框架。其主要难点在于数据的预处理和集成,
本项目计划使用简单的多层感知机模型,中间通过Dropout层防止过拟合。在设计感知机的神经网络层数的时候,需
进行多次实验之后才能确定每层的神经元数量。对于每层网络的激活函数采取目前效果最好的relu激活函数,在训练过程中
梯度下降采用Adam策略。

\subsubsection{模型评价与测试}
对模型的评价使用AUC指标(Area Under Curve)。AUC指标是目前最具有代表性也比较适配机器学习模型评测标准的指标,
通过AUC指标,我们可以通过调整门槛值(Threshold)来调整准确率和召回率。
在模型的测试阶段,我们希望通过划分测试集(Test Dataset)、验证集合(Validation Dataset)和训练集合(Training Dataset)。
在模型训练过程中,每一次优化迭代结束后用验证集合进行测试,及时调整学习率。在模型训练完成之后
使用测试集进行前向传播,绘制ROC曲线,得出AUC值,利用AUC的高低对模型总体预测精度进行评价。

在训练完成模型之后挑选几种特定的药物和多组学数据(Out-of-distribution)进行重定位研究,并进行分析讨论,综合评定
模型的效能。


\subsection{可行性分析}

\subsubsection{数据来源分析}
单细胞数据库CDRP和GDSC是目前学术界比较成熟的两个药物单细胞组学数据库,收录了较为齐全的单细胞多组学数据,
包括基因表达数据、基因拷贝数变异数据以及表观遗传学数据等。同时这些数据库也提供了药物-单细胞相互作用的数据,
并且开放下载。对于目前常见的一些药物候选化合物,PubChem数据库(有机小分子生物活性数据)也提供了相关信息的查询,
我们可以利用到上述数据库进行分析建模。


\subsubsection{模型可行性}
目前关于利用深度学习模型和单细胞组学数据、药物组学数据进行药物重定位相关的研究有很多。
其中包括利用多组学数据进行研究,


\subsubsection{个人能力自述}
本人在本科期间曾经参与过一个计算机视觉组的相关工作,有相关的深度学习经验。并且
在实验过程中用Python语言处理过相关数据,具备一定的深度学习建模和数据预处理经验,对于
模型输入多组学数据有过一定的动手操作经验,故此具备一定能力来进行该项目。
