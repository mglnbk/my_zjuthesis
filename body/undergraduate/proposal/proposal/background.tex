\section{问题提出的背景}

% \par 正文格式与具体要求\cite{zjuthesisrules}

\subsection{背景介绍}

\par \textbf{药物重定位}:
药物重定位是目前药物研发(Drug Research and Development)一个重要的研究方向,而在
如今这个计算力飞速发展,数据海量产生的时代,
利用大数据技术进行药物重定位研究是一个非常具有前景的方向。
更进一步说,如今深度学习技术的发展促使了人们对于生物问题认知态度的转变,
最明显的莫过于AlphaFold的发布,深刻影响了人们对与蛋白质结构预测的认识。
也因为基于计算方法的低成本和高效率,关于重定位的研究渐渐转向于如何更有效
地应用各种计算模型。

\par \textbf{注意力机制和Transformer架构}:
当我们读一句话时候,我们的大脑会首先会处理重要的信息,我们的“注意力”会主要集中在
语段的重要部分,基于这样的机制,设计了目前最常见的基于Key, Query, Value的注意力机制。
注意力机制可以从大量信息中聚焦挑选出最重要的部分。而自注意力机制是自注意力机制的变种,
更擅长捕捉特征内部的相关性。Transformer架构是一种采取了自注意力机制的深度学习模型
,其可以将输入数据中进行不同权重的划分。由于其强大的预测性能,Transformer被广泛
应用于计算机视觉和自然语言处理领域,比如BERT模型和GPT模型。而在生物领域,即便也有很多
Transformer架构的身影,但是由于生物学领域的问题多样繁杂,使得应用手段适用范围有限。
而生物学问题不同于计算机领域,需要很强的可解释性,
故此怎样高效地将前沿的深度学习模型应用于生物学问题充满了挑战和机遇。

\par \textbf{多模态模型的提出}:
多模态模型理论(Multi-Modal Machine Learning)是近年发展起来的一个深度学习类型。
模态指的是每一种新的数据来源或者形式。比如在基因组研究中,各种基因测序技术所产生的
不同的测序数据都是应用于不同的研究目的。比如ChIP-seq用于探明组蛋白修饰情况而DNase-seq
则是用于探明染色质开放区域。多模态学习即结合这几种测序数据来综合进行模型的训练与预测。
更进一步,上述的多模态学习可以被归类为多模态表征学习,利用多模态数据的互补性和冗余性来
处理数据的异质性。在计算机领域,多模态的概念更加广泛,而在生物学领域,多模态数据则出现的更加
普遍,且多用于综合的表征学习。故此利用多模态模型去处理异源生物医学数据需要学界进一步的
探索和尝试。


\subsubsection{项目提出的原因}
1. 数据集成问题在生物信息领域一直是一个值得关注的问题,海量生物医学数据的异质性和复杂性
促使了诸多计算生物学家和生物信息领域的研究者们对与这个领域产生了浓厚的兴趣。
2. 数据集成不仅能够充分利用到各种不同格式的数据同时也可以为模型的可解释性提供一定的启示。
3. 药物重定位的背景下进行数据集成方法研究还未被充分讨论,而结合不同来源的数据对
重定位的准确率有着明显的提升。
4. 深度学习在数据集成性能方面表现优异,具有很大的挖掘价值。
5. 目前充足的细胞系基因组数据和药物结构信息为深度学习模型提供了数据基础,也方便了项目的开展。


\subsection{本研究的意义和目的}
本研究的主要目的在于尝试利用多模态深度学习模型去解决药物重定位的数据集成问题,通过建立Transoformer多通道神经
网络模型,利用病人的多模态基因型数据和相关药物的结构信息来实现药物有效性的预测,从而实现药物重定位(药物复用)的
目的。该项目不仅可以提出一个模型为药物复用提供方法,而且也可以为今后的数据集成研究提供一个新的发展方向。
